% Boilerplate from: https://github.com/jdavis/latex-homework-template/blob/master/homework.tex
\documentclass[table]{article}
\usepackage[table]{xcolor}
\usepackage{amssymb}
\usepackage{fancyhdr}
\usepackage{extramarks}
\usepackage{amsmath}
\usepackage{amsthm}
\usepackage{amsfonts}
\usepackage{tikz}
\usepackage{graphicx}
\usepackage{enumitem}
\usepackage{logicproof}

\topmargin=-0.45in
\evensidemargin=0in
\oddsidemargin=0in
\textwidth=6.5in
\textheight=9.0in
\headsep=0.25in

\linespread{1.1}
\pagestyle{fancy}
\lhead{\hmwkAuthorName}
\chead{\hmwkTitle}
\rhead{}
\cfoot{\thepage}

\renewcommand\headrulewidth{0.4pt}
\renewcommand\footrulewidth{0.4pt}

\setcounter{secnumdepth}{0}

\newcommand{\hmwkTitle}{Final Exam}
\newcommand{\hmwkDueDate}{23 July 2020}
\newcommand{\hmwkClass}{Discrete Structures}
\newcommand{\hmwkClassTime}{Section 201}
\newcommand{\hmwkClassInstructor}{Professor Jensen}
\newcommand{\hmwkAuthorName}{\textbf{Brian Ton}}

\title{
    \vspace{2in}
    \textmd{\textbf{\hmwkClass:\ \hmwkTitle}}\\
    \normalsize\vspace{0.1in}\small{Due\ on\ \hmwkDueDate}\\
    \vspace{0.1in}\large{\textit{\hmwkClassInstructor\ \hmwkClassTime}}
    \vspace{3in}
}

\author{\hmwkAuthorName}
\date{}

\newcolumntype{g}{>{\columncolor{yellow!20}}c}

\begin{document}
\maketitle
\pagebreak
\section{Problem 1}
Write the contrapositive of ``If $a \nmid b$ and $a \mid c$, then $a \nmid (b + c)$.''
\subsection{Solution}
If $a \mid (b+c)$ then $a \mid b$ or $a \nmid c$.
\subsubsection{Explanation}
Let $p$ be the statement $a \nmid b$, $q$ be the statement $a \mid c$, and $r$ be the statement $a \nmid (b+c)$.\\
Symbolically, the original statement can thus be written as $(p \land q) \rightarrow r$. The contrapositive would then be $\neg r \rightarrow \neg (p \land q)$ which by De Morgan's law would be $\neg r \rightarrow (\neg p \lor \neg q)$, which in English would be ``If $a \mid (b+c)$ then $a \mid b$ or $a \nmid c$.''
\section{Problem 2}
Consider the function $f$ from set $X$ to set $Y$. Write the negation of the following definition for $f$ to
be one-to-one.\\
$\forall x_1, x_2 \in X$, if $f(x_1)=f(x_2)$, then $x_1=x_2$.
\subsection{Solution}
$\exists x_1, x_2 \in X$, such that $f(x_1)=f(x_2)$ and $x_1 \neq x_2$.
\subsubsection{Explanation}
Writing the original statement in full symbolic form, it becomes $\forall x_1, x_2 \in X, (f(x_1)=f(x_2) \rightarrow x_1=x_2)$.
Note that we use logical equivalences to find the negation.\\
\begin{align*}
\neg \forall x_1, x_2 \in X, (f(x_1)=f(x_2) \rightarrow x_1=x_2)
&\equiv \exists x_1, x_2 \in X, \neg (f(x_1)=f(x_2) \rightarrow x_1=x_2) && \text{(De Morgan's Law)}\\
&\equiv \exists x_1, x_2 \in X, (f(x_1)=f(x_2) \land \neg (x_1=x_2)) && \text{$(\neg (p \rightarrow q) \equiv p \land \neg q)$}\\
&\equiv \exists x_1, x_2 \in X, (f(x_1)=f(x_2) \land (x_1 \neq x_2))\\
\end{align*}
Putting this back to match the original statement, $\exists x_1, x_2 \in X$, such that $f(x_1)=f(x_2)$ and $x_1 \neq x_2$.
\section{Problem 3}
$\sqrt{3}$ is irrational. Use proof by contradiction to prove the statement ``$4\sqrt{3} - 7$ is irrational.''
\subsection{Solution}
\textbf{Proof By Contradiction}
Assume $4\sqrt{3} - 7$ is rational.\\
By definition of a rational number, $4\sqrt{3} - 7 = \frac{p}{q}$ for some integers $p$ and $q$. Hence, we can write $4\sqrt{3} = \frac{p}{q} + 7$. This would mean that $4\sqrt{3} = \frac{p+7q}{q}$. Then, $\sqrt{3} = \frac{p+7q}{4q}$. Note that by closure of the integers under addition and multiplication, $p+7q$ and $4q$ is an integer. However, this would mean that $\sqrt{3}$ can be written as the ratio of two integers, which would mean that it is rational by the definition of a rational number. This is a contradiction, by the given information. Hence, the initial assumption that $4\sqrt{3} - 7$ is rational is false, meaning that $4\sqrt{3} - 7$ is irrational.
\section{Problem 4}
A sequence $m_1, m_2, m_3, ...$ is defined by letting $m_1 = 1$ and $m_k = 2m_{k-1} + 1$, for all integers $k \geq 2$. Use induction to show that $m_n = 2^n - 1$, for all integers $n \geq 1$.
\subsection{Solution}
\textbf{Proof By Induction}\\
\textbf{Base Case:} Show that $m_1 = 2^n-1$ for $n \geq 1$. By definition, $m_1 = 1$. Note that for $n=1$, $2^1-1=2-1=1=m_1$. This completes the basis step.\\
\textbf{Inductive Hypothesis:} Assume that $m_k = 2^k-1$ for some fixed integer $k \geq 1$.\\
\textbf{Inductive Step:} By definition of the sequence:
\begin{align*}
m_{k+1}
&= 2m_k+1\\
&= 2(2^k-1)+1 && \text{(Inductive Hypothesis)}\\
&= 2^{k+1} - 2 + 1\\
&= 2^{k+1} - 1
\end{align*}
Hence, by induction, $m_n = 2^n-1$.
\section{Problem 5}
\begin{enumerate}[nosep,label=\alph*)]
\item State the Well-Ordering Principle for the integers.
\item Does the set of all positive real numbers have a least element? If not, explain why the well-ordering principle is not violated.
\item Does the set of all nonnegative integers $n$ such that $n^2 < n$ have a least element? If not, explain why the well-ordering principle is not violated.
\end{enumerate}
\subsection{Solution}
\begin{enumerate}[nosep,label=\alph*)]
\item Every nonempty set of nonnegative integers has a least element.
\item No, the set of positive real numbers does not have a least element. For instance, let $x$ be the smallest positive real number. However, note that $\frac{x}{2}$ is a positive real number and $\frac{x}{2} < x$, which means that there can be no smallest positive real number. This does not violate the well-ordering property, as the property's definition states that it applies to integers, not to real numbers.
\item There does not exist a least element in the set of all nonnegative integers $n$ such that $n^2 < n$. This is because $n^2<n$ only for real numbers $0 < n < 1$. Hence, the set will be empty, since there are no integers between $0$ and $1$. This does not violate the well-ordering property since the well ordering property only applies to nonempty sets.
\end{enumerate}
\end{document}