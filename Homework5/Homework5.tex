% Boilerplate from: https://github.com/jdavis/latex-homework-template/blob/master/homework.tex
\documentclass[table]{article}
\usepackage[table]{xcolor}
\usepackage{amssymb}
\usepackage{fancyhdr}
\usepackage{extramarks}
\usepackage{amsmath}
\usepackage{amsthm}
\usepackage{amsfonts}
\usepackage{tikz}
\usepackage{graphicx}
\usepackage{enumitem}
\usepackage{logicproof}

\topmargin=-0.45in
\evensidemargin=0in
\oddsidemargin=0in
\textwidth=6.5in
\textheight=9.0in
\headsep=0.25in

\linespread{1.1}
\pagestyle{fancy}
\lhead{\hmwkAuthorName}
\chead{\hmwkTitle}
\rhead{}
\cfoot{\thepage}

\renewcommand\headrulewidth{0.4pt}
\renewcommand\footrulewidth{0.4pt}

\setcounter{secnumdepth}{0}

\newcommand{\hmwkTitle}{Homework\ \#5}
\newcommand{\hmwkDueDate}{17 July 2020}
\newcommand{\hmwkClass}{Discrete Structures}
\newcommand{\hmwkClassTime}{Section 201}
\newcommand{\hmwkClassInstructor}{Professor Jensen}
\newcommand{\hmwkAuthorName}{\textbf{Brian Ton}}

\title{
    \vspace{2in}
    \textmd{\textbf{\hmwkClass:\ \hmwkTitle}}\\
    \normalsize\vspace{0.1in}\small{Due\ on\ \hmwkDueDate}\\
    \vspace{0.1in}\large{\textit{\hmwkClassInstructor\ \hmwkClassTime}}
    \vspace{3in}
}

\author{\hmwkAuthorName}
\date{}

\newcolumntype{g}{>{\columncolor{yellow!20}}c}

\begin{document}
\maketitle
\pagebreak
\section{Problem 1}
Let the universal set be the set $\{a, b, c, d, e, f, g\}$ and let $A = \{a, d, f\}$ and $B = \{d, g\}$.
\begin{enumerate}[nosep, label=\alph*)]
\item Find $A \cup B$.
\item Find $A \cap B$.
\item Find $A - B$.
\item Find $B - A$.
\item Find $\overline{A}$. Note that $\overline{A} = U - A$
\item Find $A \times B$.
\item Find $\mathcal{P}(B)$.
\end{enumerate}
\subsection{Solution}
\begin{enumerate}[nosep, label=\alph*)]
\item $A \cup B = \{a, d, f, g\}$
\item $A \cap B = \{d\}$.
\item $A - B = \{a, f\}$.
\item $B - A = \{g\}$.
\item $\overline{A} = \{b, c, e, g\}$
\item $A \times B = \{(a, d), (d, d), (f, d), (a, g), (d, g), (f, g)\}$.
\item $\mathcal{P}(B) = \{\emptyset, \{d\}, \{g\}, \{d, g\}\}$.
\end{enumerate}
\section{Problem 2}
Consider the set $S = \{1, 2, 3, 4, 5, 6\}$. Answer each question \textit{Yes} or \textit{No}.
\begin{enumerate}[nosep, label=\alph*)]
\item Is $\{\{1, 4, 5\}, \{2, 3\}, \{2, 6\}\}$ a partition of $S$?
\item Is $\{\{1, 2, 5\}, \{3\}, \{4, 6\}\}$ a partition of $S$?
\item Is $\{\{1,4\}, \{2,3\}, \{6\}\}$ a partition of $S$?
\end{enumerate}
\subsection{Solution}
\begin{enumerate}[nosep, label=\alph*)]
\item \textit{No}
\item \textit{Yes}
\item \textit{No}
\end{enumerate}
\subsubsection{Explanation}
\textbf{Part A}\\
Since not all elements within the proposed partition are disjoint (i.e. the sets $\{2, 3\}$ and $\{2, 6\}$), it cannot be a partition of $S$.\\
\textbf{Part B}\\
Since all elements within the proposed partition are disjoint and the union of them is equal to $S$, it is a partition of $S$.\\
\textbf{Part C}\\
Since $5$ is not in any of the elements within the proposed partition, the union of the elements is not equal to $S$.
\section{Problem 3}
Suppose $A$ is a set with 8 elements. What is $|\mathcal{P}(A)|$?
\subsection{Solution}
$|\mathcal{P}(A)|=2^8=256$
\subsubsection{Explanation}
Note that $|\mathcal{P}(A)| = 2^{|A|}$.\\
This can be proven using induction.\\
\textbf{Base Case:} Show $|\mathcal{P}(\emptyset)|=2^{|\emptyset|}$.\\
$|\emptyset|=0$\\
$|\mathcal{P}(\emptyset)| = |\{\emptyset\}| = 1$\\
$2^{|\emptyset|} = 2^0 = 1 = |\mathcal{P}(\emptyset)|$.\\
\textbf{Induction Hypothesis:} Assume for some fixed $k \geq 0, \forall$ sets $S$, if $|S|=k$, then $|\mathcal{P}(s)|=2^k$.\\
\textbf{Induction Step:} Let $T$ be a set with $|T| = k+1$. Let $S = T - \{a\}$ for an arbitrary element $a$ in $T$. Thus, $|S| = k$. Let $B$ denote the set that contains all the elements in $\mathcal{P}(S)$ with $a$ adjoined to them. Here, note that $|\mathcal{P}(S)| = |B|$, since the construction of $B$ preserves the equality of cardinality. Then, note that $\mathcal{P}(T) = \mathcal{P}(S) \cup B$. Hence $|\mathcal{P}(T)| = |\mathcal{P}(S)| + |B|$. Since $|\mathcal{P}(S)| = |B|$, $|\mathcal{P}(T)| = 2 \cdot |\mathcal{P}(S)|$. By the inductive step, since $|\mathcal{P}(S)| = 2^k$, $|\mathcal{P}(T)| = 2 \cdot 2^k = 2^{k+1} = 2^{|T|}$.\\
$\therefore$ By induction, $|\mathcal{P}(A)| = 2^{|A|}$ for some arbitrary set $A$.
\section{Problem 4}
Define sets $A$ and $B$ as follows:\\
\indent $A = \{m \in \mathbb{Z} \mid m = 3a \;\text{for some integer a}\}$\\
\indent $B = \{n \in \mathbb{Z} \mid n = 3b - 3 \;\text{for some integer b}\}$\\
Show $A = B$.
\subsection{Solution}
\subsubsection{Part 1}
\textbf{Show:} $A \subseteq B$.\\
Let $x \in A$. By definition of $A$, $x = 3a$ for some $a \in \mathbb{Z}$. Let $b = a + 1$. Note that by closure under addition, $b$ must be an integer. By substitution, $3b - 3 = 3(a + 1) - 3 = 3a + 3 - 3 = 3a = x$. Hence, $x = 3b - 3$ for some $b \in \mathbb{Z}$ (namely $b = a + 1$). In other words, $x \in B$ and $A \subseteq B$.
\subsubsection{Part 2}
\textbf{Show:} $B \subseteq A$.\\
Let $x \in B$. By definition of $B$, $x = 3b - 3$ for some $b \in \mathbb{Z}$. Let $a = b - 1$. Note that by closure under addition, $a$ must be an integer. By substitution, $3a = 3(b - 1) = 3b - 3 = x$. Thus, $x = 3a$ for some $a \in \mathbb{Z}$ (namely $a = b - 1$). In other words, $x \in A$ and $B \subseteq A$.\\~\\
$\therefore$ Since $A \subseteq B$ and $B \subseteq A$, $A = B$.
\section{Problem 5}
Prove the statement using the subset method if it is true and find a counterexample if it is
false. Assume all sets are subsets of a universal set $U$.
\begin{enumerate}[nosep, label=\alph*)]
\item For all sets $A$, $B$, and $C$, $A - (B - C) = (A - B) - C$.
\item For all sets $A$ and $B$, $A \cap (A \cup B) = A$.
\end{enumerate}
\subsection{Solution}
\begin{enumerate}[nosep, label=\alph*)]
\item False
\item True
\end{enumerate}
\subsubsection{Part A}
Let $A = \{1, 2, 3, 4, 5\}$, $B = \{2, 3\}$, and $C = \{3\}$.\\
Here, $A - (B - C) = A - (\{2\}) = \{1, 3, 4, 5\}$.\\
Additionally, $(A - B) - C = (\{1, 4, 5\}) - C = \{1, 4, 5\}$.\\
Note that $A - (B - C) = \{1, 3, 4, 5\} \neq (A - B) - C = \{1, 4, 5\}$. Hence, the statement is false.
\subsubsection{Part B}
\textbf{Part 1}\\
\textbf{Show:} $A \cap (A \cup B) \subseteq A$.\\
Let $x \in A \cap (A \cup B)$. By definition of intersection, $x \in A$ and $x \in (A \cup B)$. Therefore, by simplification, $x \in A$ and thus $A \cap (A \cup B) \subseteq A$.\\~\\
\textbf{Part 2}\\
\textbf{Show:} $A \subseteq A \cap (A \cup B)$.\\
Let $x \in A$. By definition of union, if an element is in $A \cup B$, then the element must be in $A$ or $B$. Hence, since $x \in A$, $x \in (A \cup B)$. Furthermore, by definition of an intersection, if an element is in both $A$ and $B$, then it must also be in $A \cap B$. Since $x \in A$ and $x \in (A \cup B)$, $x \in A \cap (A \cup B)$ and thus $A \subseteq A \cap (A \cup B)$.\\~\\
$\therefore$ Since $A \cap (A \cup B) \subseteq A$ and $A \subseteq A \cap (A \cup B)$, $A \cap (A \cup B) = A$.
\section{Problem 6}
State De Morgan's Laws for sets.
\subsection{Solution}
\begin{enumerate}[nosep]
\item The complement of the union of two sets is the intersection of the complements of the two sets.
\item The complement of the intersection of two sets is the union of the complements of the two sets.
\end{enumerate}
\end{document}
