% Boilerplate from: https://github.com/jdavis/latex-homework-template/blob/master/homework.tex
\documentclass[table]{article}
\usepackage[table]{xcolor}
\usepackage{amssymb}
\usepackage{fancyhdr}
\usepackage{extramarks}
\usepackage{amsmath}
\usepackage{amsthm}
\usepackage{amsfonts}
\usepackage{tikz}
\usepackage{graphicx}
\usepackage{enumitem}
\usepackage{logicproof}

\topmargin=-0.45in
\evensidemargin=0in
\oddsidemargin=0in
\textwidth=6.5in
\textheight=9.0in
\headsep=0.25in

\linespread{1.1}
\pagestyle{fancy}
\lhead{\hmwkAuthorName}
\chead{\hmwkTitle}
\rhead{}
\cfoot{\thepage}

\renewcommand\headrulewidth{0.4pt}
\renewcommand\footrulewidth{0.4pt}

%\setlength\parindent{15pt}%

\newcommand{\enterProblemHeader}[1]{
    \nobreak\extramarks{}{Problem \arabic{#1} continued on next page\ldots}\nobreak{}
    \nobreak\extramarks{Problem \arabic{#1} (continued)}{Problem \arabic{#1} continued on next page\ldots}\nobreak{}
}

\newcommand{\exitProblemHeader}[1]{
    \nobreak\extramarks{Problem \arabic{#1} (continued)}{Problem \arabic{#1} continued on next page\ldots}\nobreak{}
    \stepcounter{#1}
    \nobreak\extramarks{Problem \arabic{#1}}{}\nobreak{}
}

\setcounter{secnumdepth}{0}

\newcommand{\hmwkTitle}{Homework\ \#4}
\newcommand{\hmwkDueDate}{13 July 2020}
\newcommand{\hmwkClass}{Discrete Structures}
\newcommand{\hmwkClassTime}{Section 201}
\newcommand{\hmwkClassInstructor}{Professor Jensen}
\newcommand{\hmwkAuthorName}{\textbf{Brian Ton}}

\title{
    \vspace{2in}
    \textmd{\textbf{\hmwkClass:\ \hmwkTitle}}\\
    \normalsize\vspace{0.1in}\small{Due\ on\ \hmwkDueDate}\\
    \vspace{0.1in}\large{\textit{\hmwkClassInstructor\ \hmwkClassTime}}
    \vspace{3in}
}

\author{\hmwkAuthorName}
\date{}

\newcommand{\solution}{\textbf{\large Solution}}

\newcolumntype{g}{>{\columncolor{yellow!20}}c}
\begin{document}
\maketitle
\pagebreak
\section{Problem 1}
Prove or disprove the statement ``The set of integers is closed under division.''
\subsection{Solution}
This statement is false.
\subsubsection{Proof by Counterexample}
Consider $1 \div 2$.\\
Note that $1$ and $2$ are both integers. Since the quotient of these two integers is not also an integer, the integers cannot be closed under division.
\section{Problem 2}
Is the statement ``$\forall x,y \in \mathbb{R}, \sqrt{x + y} = \sqrt{x} + \sqrt{y}$'' True or False? If it is false, provide a counterexample.
\subsection{Solution}
This statement is false.
\subsubsection{Proof by Counterexample}
Consider $x=9$ and $y=16$. In this case, note that $\sqrt{x+y} = \sqrt{9+16} = \sqrt{25} = 5$ and that $\sqrt{x} + \sqrt{y}= \sqrt{9} + \sqrt{16} = 3 + 4 = 7$.\\
$\therefore$ The statement $\forall x,y \in \mathbb{R}, \sqrt{x + y} = \sqrt{x} + \sqrt{y}$ is not true.
\section{Problem 3}
Is the statement ``$\exists x,y \in \mathbb{R}, \sqrt{x + y} = \sqrt{x} + \sqrt{y}$'' True or False? If it is true, provide an example.
\subsection{Solution}
This statement is true.
\subsubsection{Proof by Example}
Consider $x=9$ and $y=0$. In this case, note that $\sqrt{x+y} = \sqrt{9+0} = \sqrt{9} = 3$ and that $\sqrt{x} + \sqrt{y}= \sqrt{9} + \sqrt{0} = 3 + 0 = 3$. In fact, this is true for any $x$, provided that $y=0$.
\subsubsection{Proof by Direct Proof (because I think it's interesting)}
Prove that for $x \in \mathbb{R}$ and $y=0$, $\sqrt{x + y} = \sqrt{x} + \sqrt{y}$.\\
\textbf{Proof:} Let $x$ be an arbitrary real number and $y=0$. Note that  $0$ is a solution to $x^2=0$ (since $0^2=0\cdot0=0$) and that the square root $r$ of a number $x$ is defined as a non-negative real number that satsifies $r^2=x$. From this, we can conclude that $\sqrt{0}=0$. This means that we can write $\sqrt{x+y}$ as $\sqrt{x+0}=\sqrt{x}$, since $y=0$ and $x+0=x$ for all real numbers $x$ by the Additive Identity Axiom. Continuing, note that $\sqrt{x} = \sqrt{x} + 0$ by the Additive Identity Axiom and that since $0=\sqrt{0}$, then we can write $\sqrt{x}+0=\sqrt{x}+\sqrt{0}$. Since this is in the form $\sqrt{x} + \sqrt{y}$ where $x$ is a real number and $y=0$, the statement that for $x \in \mathbb{R}$ and $y=0$, $\sqrt{x + y} = \sqrt{x} + \sqrt{y}$ holds.\\
This result then implies the existence of $x,y \in \mathbb{R}$ such that $\sqrt{x + y} = \sqrt{x} + \sqrt{y}$, meaning that the original statement is true.
\section{Problem 4}
Use the definition of even and odd to prove that for all integers $m$, if $m$ is even then $3m + 5$
is odd.
\subsection{Solution}
\subsubsection{Proof By Direct Proof}
\textbf{Proof:} Let $m$ be an even integer. By definition of an even integer, $m=2k$ for some integer $k$. It follows, then, that $3m + 5 = 3(2k) + 5 = 6k + 5 = 6k + 4 + 1 = (6k + 4) + 1 = 2(3k + 2) + 1$. Because integers are closed under multiplication and addition, $3k + 2$ must be an integer and, again because of closure of the integers under multiplication and addition, $2(3k+2)+1$ is also an integer. By definition of an odd integer, $2(3k+2) + 1$ is an odd integer and thus that $3m + 5$ is odd. This means then, that if $m$ is an even integer, then $3m + 5$ is odd.
\section{Problem 5}
Use the definition of even and odd to prove that if $k$ is any even integer and $m$ is any odd
integer, then $k^2 + m^2$ is odd.
\subsection{Solution}
\subsubsection{Proof By Direct Proof}
\textbf{Proof:} Let $k$ be an even integer and $m$ be an odd integer. By definition of odd and even integers, $k=2x$ and $m=2y+1$ for some integers $x$ and $y$. It follows, then, that $k^2 + m^2 = (2x)^2 + (2y+1)^2 = 4x^2 + 4y^2 + 4y + 1 = 2(2x^2 + 2y^2 + 2y) + 1$. By closure of the integers under multiplication, it follows that $2x^2 + 2y^2 + 2y$ is an integer and, by closure of the integers under multiplication and addition, that $2(2x^2 + 2y^2 + 2y) + 1$ is an integer. By definition of an odd integer, $2(2x^2 + 2y^2 + 2y) + 1$ must be odd. This means then, that if $k$ is an even integer and $m$ is an odd integer, then $k^2 + m^2$ must be odd.
\section{Problem 6}
Use the definition of even and odd to prove that the product of any two consecutive
integers is even.
\subsection{Solution}
\subsubsection{Proof by Direct Proof}
\textbf{Lemma:} An arbitrary integer $n$ is either even or odd.\\
\textbf{Proof:} By the Division Algorithm (Quotient Remainder Theorem), $n$ can be written with unique integers $q$ and $r$ such that $n = 2q + r$, with $0 \leq r < 2$. Since $r$ must be an integer, $r$ can be only the values $0$ or $1$. If $r$ is 0, then $n=2q$, which means that $n$ is even, by the definition of an even integer. If $r$ is $1$, then $n=2q+1$, which means $n$ must be an odd integer. Thus, an arbitrary integer $n$ is either even or odd.\\~\\
\textbf{Proof:} Let $n$ and $n+1$ be two consecutive integers. By Lemma 1, $n$ has to be either even or odd.\\
\textbf{Case 1:} Assume $n$ is an even integer. By definition of an even integer, $n = 2k$ for some integer $k$. Additionally, $n+1=2k+1$. It follows that $n(n+1)=(2k)(2k+1)=4k^2+2k=2(2k^2+k)$. By closure of integers under multiplication and addition, $2k^2+1$ is an integer and thus $2(2k^2+k)$ is also an integer. By definition of an even integer, $2(2k^2+k)$ is even, thus meaning that $n(n+1)$ is even if $n$ is also even.\\
\textbf{Case 2:} Assume $n$ is an odd integer. By definition of an odd integer, $n = 2k+1$ for some integer k. Additionally, $n+1=2k+1+1=2k+2$. It follows that $n(n+1)=(2k+1)(2k+2)=4k^2+6k+2=2(2k^2+3k+1)$. By closure of integers under addition and multiplication, $2k^2+3k+1$ is an integer and thus $2(2k^2+3k+1)$ is also an integer. By definition of an even integer, $2(2k^2+3k+1)$ is even, thus meaning that $n(n+1)$ is even if $n$ is odd.\\
Since $n(n+1)$ is even, regardless of the parity of $n$, the product of two consecutive integers is always even.
\section{Problem 7}
Prove by contrapositive ``For all integers $a$, $b$, and $c$, if $a \mid b$ and $a \nmid c$, then $a \nmid (b + c)$.''
\large
\subsection{Solution}
\subsubsection{Proof By Contrapositive}
Note that the contrapositive of the statement is ``For all integers $a$, $b$, and $c$, if $a \mid (b + c)$, then $a \nmid b$ or $a \mid c$.'' Additionally, note that:
\begin{align*}
p \rightarrow (q \lor r) &\equiv \neg p \lor q \lor r && \text{(Conditional Disjunction Equivalence)}\\
&\equiv (\neg p \lor q) \lor r && \text{(Associative Law)}\\
&\equiv \neg (\neg p \lor q) \rightarrow r && \text{(Conditional Disjunction Equivalence)}\\
&\equiv (p \land \neg q) \rightarrow r && \text{(De Morgan's Law)}
\end{align*}
$\therefore$ It suffices to show that for all integers $a$, $b$, and $c$, if $a \mid (b+c)$ and $a \mid b$, then $a \mid c$.\\
\textbf{Proof:} Let $a$, $b$, and $c$ be integers such that $b+c$ and $b$ are both divisible by $a$. By definition of divisibility, $b + c = na$ and $b = ma$ for some integers $n$, $m$. By substitution, $b + c = na$ can be rewritten as $ma + c = na$. Thus, $c = na - ma = a(n - m)$. Because of closure of integers under addition, $n - m$ is an integer, and by the definition of divisibility, $c$ is divisible by $a$. Thus, since for all integers $a$, $b$, and $c$, if $a \mid (b+c)$ and $a \mid b$, then $a \mid c$ is true, the contrapositive must be true, and thus the original statement must be true.
\subsubsection{Direct Proof (because I think it's more intuitive than the proof by contraposition)}
Note that this proof relies on a result (the Division Algorithm, also known as the Quotient-Remainder Theorem) from Chapter 4.1.\\
Prove that for all integers $a$, $b$, and $c$, if $a \mid b$ and $a \nmid c$, then $a \nmid (b + c)$.\\~\\
\textbf{Lemma 1: If $a$ and $b$ are integers, and $a+b+c$ is an integer, then $c$ must be an integer.}\\
\textbf{Proof:} Let $m=a+b+c$ where $m$ is an integer. Additionally, let $a$ and $b$ be integers. Starting with the definition of $m$, we can write that $a+b+c=m$. By closure of integers under subtraction, note that $m-a-b$ is an integer. However, since $m-a-b=c$, $c$ must also be an integer.\\
$\therefore$ If $a$ and $b$ are integers, and $a+b+c$ is an integer, then $c$ must be an integer.\\~\\
\textbf{Proof:} Let $a$, $b$, and $c$ be integers such that $b$ is divisible by $a$ but $c$ is not divisible by $a$. Note that by the definition of divisibility, since $a \mid b$, there exists an integer $q_1$ such that $b = q_1a$. Additionally, by the Division Algorithm since $a \nmid c$, then there exists unique nonzero integers $q_2$ and $r$ such that $c = q_2a + r$. Then, $b+c=q_1a+q_2a+r=a(q_1+q_2)+r$. However, note that $a(q_1+q_2)+r$ cannot be divisible by $a$. If $a(q_1+q_2)+r$ were divisible by $a$, then $a(q_1+q_2)+r=a(q_1+q_2+\frac{r}{a})$ where $\frac{r}{a}$ must be an integer by Lemma 1 since $q_1$ and $q_2$ are integers and $q_1+q_2+\frac{r}{a}$ must be an integer by the definition of divisibility. However, this would contradict the fact that $a \nmid c$, since it would follow that $r$ is divisible by $a$ and that $c=(q_2+\frac{r}{a})a$ (where $c$ is divisible by $a$ because $q_2$ and $\frac{r}{a}$ are integers and due to closure of the integers under addition, $q_2+\frac{r}{a}$ is an integer, and therefore, by the definition of divisiblility, $c$ is divisible by $a$). Thus, $a(q_1+q_2)+r$ cannot be divisible by $a$.\\
Since $a \nmid (a(q_1+q_2)+r)$ and $a(q_1+q_2)+r = (b+c)$, $a \nmid (b+c)$.\\
$\therefore$ For all integers $a$, $b$, and $c$, if $a \mid b$ and $a \nmid c$, then $a \nmid (b + c)$.
\section{Problem 8}
Prove or disprove the statement ``The square root of an irrational number is irrational.''
\subsection{Solution}
\subsubsection{Proof by Contradiction}
Let $x$ be an irrational number.\\
Assume $\sqrt{x}$ is rational.\\
Since $\sqrt{x}$ is rational, there exist integers $p$ and $q$ such that $\sqrt{x}=\frac{p}{q}$. It then follows that $x=\frac{p^2}{q^2}$. By closure of integers under multiplication, $p^2$ and $q^2$ must be integers. However, since $x$ can be written as the ratio of integers, it must be rational. This is a contradiction, since $x$ was defined to be an irrational number, and thus the assumption that $\sqrt{x}$ is rational is false.\\
$\therefore$ The square root of an irrational number is irrational.
\section{Problem 9}
Is the statement ``$5\sqrt{2} - 3$ is irrational'' True or False? Prove the statement if it is True.
Disprove the statement if it is False. (You can use the fact that $\sqrt{2}$ is irrational)
\subsection{Solution}
\subsubsection{Proof By Contradiction}
\textbf{Proof:} Assume that $5\sqrt{2} - 3$ is rational.\\
If $5\sqrt{2} - 3$ is rational, then there exist integers $p$ and $q$ such that $5\sqrt{2} - 3 = \frac{p}{q}$. Adding $3$ to both sides of the equation gives $5\sqrt{2} = \frac{p}{q} = \frac{p+3q}{q}$. Then, dividing both sides of the equation by 5 gives $\sqrt{2} = \frac{p+3q}{5q}$. By closure of integers under addition and multiplication, $p+3q$ and $5q$ must be integers. However, since $\sqrt{2}$ can be written as the ratio of two integers, it must be rational. This leads to a contradiction, since $\sqrt{2}$ is irrational. Hence, the assumption that $5\sqrt{2} - 3$ is rational is false.\\
$\therefore 5\sqrt{2} - 3$ must be irrational.
\section{Problem 10}
Prove the statement is false. ``There is a positive integer $n$ such that $n^2 + 6n + 5$ is prime.''
\subsection{Solution}
\subsubsection{Direct Proof}
\textbf{Proof:} Let $m$ be an integer of the form $n^2 + 6n + 5$ where $n$ is a positive integer.\\
Note that $n^2 + 6n + 5 = (n+1)(n+5)$. Since any $m$ has the two factors $(n+1)$ and $(n+5)$ and since $n + 1 \neq 1$ and $n + 5 \neq 1$ (because $n \geq 1$), it cannot be prime (since it has at least 2 positive factors, neither of which is 1, meaning that it does not have only the factors $1$ and itself).\\
$\therefore$ The statement ``There is a positive integer $n$ such that $n^2 + 6n + 5$ is prime'' is false.
\end{document}