% Boilerplate from: https://github.com/jdavis/latex-homework-template/blob/master/homework.tex
\documentclass[table]{article}
\usepackage[table]{xcolor}
\usepackage{amssymb}
\usepackage{fancyhdr}
\usepackage{extramarks}
\usepackage{amsmath}
\usepackage{amsthm}
\usepackage{amsfonts}
\usepackage{tikz}
\usepackage{graphicx}
\usepackage{enumitem}

\topmargin=-0.45in
\evensidemargin=0in
\oddsidemargin=0in
\textwidth=6.5in
\textheight=9.0in
\headsep=0.25in

\linespread{1.1}
\pagestyle{fancy}
\lhead{\hmwkAuthorName}
\chead{\hmwkTitle}
\rhead{}
\cfoot{\thepage}

\renewcommand\headrulewidth{0.4pt}
\renewcommand\footrulewidth{0.4pt}

%\setlength\parindent{15pt}%

\newcommand{\enterProblemHeader}[1]{
    \nobreak\extramarks{}{Problem \arabic{#1} continued on next page\ldots}\nobreak{}
    \nobreak\extramarks{Problem \arabic{#1} (continued)}{Problem \arabic{#1} continued on next page\ldots}\nobreak{}
}

\newcommand{\exitProblemHeader}[1]{
    \nobreak\extramarks{Problem \arabic{#1} (continued)}{Problem \arabic{#1} continued on next page\ldots}\nobreak{}
    \stepcounter{#1}
    \nobreak\extramarks{Problem \arabic{#1}}{}\nobreak{}
}

\setcounter{secnumdepth}{0}
\newcounter{partCounter}
\newcounter{homeworkProblemCounter}
\setcounter{homeworkProblemCounter}{1}
\nobreak\extramarks{Problem \arabic{homeworkProblemCounter}}{}\nobreak{}

\newcommand{\hmwkTitle}{Homework\ \#2}
\newcommand{\hmwkDueDate}{30 June 2020}
\newcommand{\hmwkClass}{Discrete Structures}
\newcommand{\hmwkClassTime}{Section 201}
\newcommand{\hmwkClassInstructor}{Professor Jensen}
\newcommand{\hmwkAuthorName}{\textbf{Brian Ton}}

\title{
    \vspace{2in}
    \textmd{\textbf{\hmwkClass:\ \hmwkTitle}}\\
    \normalsize\vspace{0.1in}\small{Due\ on\ \hmwkDueDate}\\
    \vspace{0.1in}\large{\textit{\hmwkClassInstructor\ \hmwkClassTime}}
    \vspace{3in}
}

\author{\hmwkAuthorName}
\date{}

\newcommand{\solution}{\textbf{\large Solution}}

\newcolumntype{g}{>{\columncolor{yellow!20}}c}

\begin{document}
\maketitle

\pagebreak
\section{Problem 1}
Is the statement ``$\forall$ real numbers $x$ and $y$, $\sqrt{x + y} = \sqrt{x} + \sqrt{y}$'' \textit{True} or \textit{False}? If it is false, provide a counterexample.
\subsection{Solution}
\textbf{False}\\
Consider the case $x=4$, $y=5$. In this case, $\sqrt{x + y} = \sqrt{4 + 5} = \sqrt{9} = 3$. However, $\sqrt{4} + \sqrt{5} = 2 + \sqrt{5} \neq 3$.
\section{Problem 2}
Is the statement ``$\exists$ real numbers $x$ and $y$ such that $\sqrt{x + y} = \sqrt{x} + \sqrt{y}$'' \textit{True} or \textit{False}? If it is true, provide an example.
\subsection{Solution}
\textbf{True}\\
Consider the case $x=1$, $y=0$. In this case, $\sqrt{x + y} = \sqrt{1 + 0} = \sqrt{1} = 1$ and $\sqrt{1} + \sqrt{0} = \sqrt{1} = 1$. More generally, this is true when at least one of $x$ or $y$ is 0 (which follows from the identity element axiom of addition for real numbers, as $\sqrt{0} = 0$ and $\sqrt{x + 0} = \sqrt{x}$, meaning that $\sqrt{x + 0} = \sqrt{x} + \sqrt{0}$).
\section{Problem 3}
Write a negation for each statement.
\begin{enumerate}[nosep, label=\alph*)]
\item There exists a real number $x$ such that $x \leq -2$
\item $\forall$ computer programs $P$, if $P$ compiles without error messages, then P is correct.
\item $\forall$ integers $n$, $\exists$ a prime number $p$ such that $n < p < 2n$.
\end{enumerate}
\subsection{Solution}
\begin{enumerate}[nosep, label=\alph*)]
\item For all real numbers $x$, $x > -2$
\item There exists a computer program $P$ such that $P$ compiles without error messages and P is not correct.
\item There exists an integer $n$ such that for all prime numbers $p$, $(p \leq x) \lor (p \geq 2x)$
\end{enumerate}
\subsubsection{Part A}
Let $P(x)$ be the statement $x \leq -2$.\\
The statement can then be written as $\exists xP(x)$. By De Morgan's Law of Quantifiers, $\neg \exists xP(x) \equiv \forall x \neg P(x)$. Additionally, note that $\neg(x \leq -2) \equiv (x > -2)$. Thus, the negation of the original statement is $\forall x(x > -2)$, which in English is ``For all real numbers $x$, $x > -2$''.
\subsubsection{Part B}
Let $E(x)$ be the statement ``$x$ compiles without error messages.'' and $C(x)$ be the statement ``$x$ is correct.''\\
The original statement can then be written as $\forall P(E(P) \rightarrow C(P))$. By De Morgan's Law of Quantifiers, $\neg \forall P(E(P) \rightarrow C(P)) \equiv \exists P \neg (E(P) \rightarrow C(P))$. By conditional disjunction, $\exists P \neg (E(P) \rightarrow C(P)) \equiv \exists P \neg (\neg E(P) \lor C(P))$, and by De Morgan's law of propositional logic, $\exists P \neg (\neg E(P) \lor C(P)) \equiv\exists P (E(P) \land \neg C(P))$, which in English would be ``There exists a computer program $P$ such that $P$ compiles without error messages and P is not correct.''
\subsubsection{Part C}
Let $P(x, p)$ be the statement $x < p < 2x$. The original statement can then be written as $\forall n \exists p(P(n, p))$. Note that $\neg(x < p < 2x) \equiv (p \leq x) \lor (p \geq 2x)$. By De Morgan's Law of Quantifiers,$\neg \forall n \exists p(P(n, p)) \equiv \exists n \neg \exists p(P(n, p)) \equiv \exists n \forall p(\neg P(n, p))$. Then, taking note of the negation of $P(x)$, $\exists n \forall p(\neg P(n, p)) \equiv \exists n \forall p((p \leq n) \lor (p \geq 2n))$. Written in English, the statement would be ``There exists an integer $n$ such that for all prime numbers $p$, $(p \leq n) \lor (p \geq 2n)$.''
\section{Problem 4}
Show that $\exists xP(x) \land \exists xQ(x)$ and $\exists x(P(x) \land Q(x))$ are not logically equivalent.
\subsection{Solution}
Let $P(x)$ be the statement $x < 1$ and $Q(x)$ be the statement $x > 1$, where $x \in \mathbb{Z}$.\\
The first statement is then $\exists x(x < 1) \land \exists x(x > 1)$. This is a true statement, as there does exist an integer x that is less than 1 (e.g. 0) and an x that is greater than 1 (e.g. 2). However, the second statement is false, as there exists no integer x that is both greater than and less than 1. $\therefore \exists xP(x) \land \exists xQ(x) \not\equiv \exists (P(x) \land Q(x))$
\section{Problem 5}
Show that $\forall xP(x) \lor \forall xQ(x)$ and $\forall x(P(x) \lor Q(x))$ are not logically equivalent.
\subsection{Solution}
Let $P(x)$ be the statement $x \geq 1$ and $Q(x)$ be the statement $x < 1$, where $x \in \mathbb{Z}$.\\
The first statement is false, since $\forall xP(x)$ is false, as not every integer is greater than or equal to 1, and $\forall xQ(x)$ is false, as not every integer is less than 1. However, the second statement is true, as every integer is greater than or equal to or less than 1. $\therefore \forall xP(x) \lor \forall xQ(x) \not\equiv \forall x(P(x) \lor Q(x))$.
\section{Problem 6}
Let $P(x,y)$ be the statement ``$xy=1$''. If the domain for both variables is the set of nonzero real numbers, what are the truth values?
\begin{enumerate}[nosep, label=\alph*)]
\item $\exists y \forall x P(x,y)$
\item $\forall x \exists y P(x,y)$
\end{enumerate}
\subsection{Solution}
\begin{enumerate}[nosep, label=\alph*)]
\item False
\item True
\end{enumerate}
\subsubsection{Part A}
Note that the statement is equivalent to saying that there exists a nonzero real $y$ that is the multiplicative inverse of all nonzero real $x$.\\
By Theorem 4 in Appendix A1.2 (also proven in the section below, as its proof is not given in the book), the multiplicative inverse of a nonzero real number $x$ is unique (i.e. there is only 1 multiplicative inverse for each nonzero real x). Thus, the only way for the original statement to be true is if all nonzero real $x$ share the same singular multiplicative inverse $y$ (a universal generalization of the fact that $y$, according to the statement, is a multiplicative inverse of an arbitrary nonzero real $x$, and by Theorem 4, the only multiplicative inverse of an arbitrary $x$).
 However, since $3$ has the (unique) multiplicative inverse of $\frac{1}{3}$ and $2$ has the (unique) multiplicative inverse $\frac{1}{2}$, it cannot be the case that there exists a $y$ that is the multiplicative inverse of all $x$, and thus the statement is false.\\
\textbf{\textit{Remark 1}}
\textit{In the above proof, Theorem 4 is needed to cover the case in which both $\frac{1}{x}$ and a constant $y$ where $\frac{1}{x} \neq y$ are multiplicative inverses of an arbitrary $x$, as then it would be the case that $y$ is a shared multiplicative inverse of all $x$ and thus the statement would be true.}\\
\textbf{\textit{Remark 2}}
\textit{After going to office hours on Tuesday, I realized that the proof can actually be done easier if done in ``reverse.'' Instead of considering the statement as ``there exists y that is the multiplicative inverse of all x,'' due to the nature of multiplicative inverses, one can actually reverse the statement, considering the statement as ``there exists a $y$ that has every real number $x$ as its multiplicative inverse.'' The result then, trivially follows from Theorem 4, as it says that all multiplicative inverses must be unique.}\\~\\
\textbf{Theorem 4 (Appendix 1.2)}
\textit{The multiplicative inverse of a nonzero real number is unique.}
\begin{proof}
Let $x$ be a nonzero real number.\\
Let $a$ and $b$ both be the multiplicative inverse of x.\\
By the inverse law of multiplication:\\
$x \cdot a = a \cdot x = 1$\\
$x \cdot b = b \cdot x = 1$\\
We can then write:\\
\begin{align*}
a &= a \cdot 1 && \text{(Multiplicative Identity Law)}\\
&= a \cdot (x \cdot b) && {\text{$(x \cdot b = b \cdot x = 1)$}}\\
&= (a \cdot x) \cdot b && {\text{(Associative Law for Multiplication)}}\\
&= 1 \cdot b && {\text{$(x \cdot a = a \cdot x = 1)$}}\\
&= b && {\text{(Multiplicative Identity Law)}}
\end{align*}
Since $a=b$, $a$ and $b$ must be the same multiplicative inverse, and thus the multiplicative inverse of a nonzero real number $x$ is unique.
\end{proof}
\subsubsection{Part B}
Note that the statement is equivalent to saying that for every nonzero real $x$, there exists a nonzero real $y$ that is its multiplicative inverse. This follows trivially from the inverse law of multiplication of the field axioms.
\end{document}